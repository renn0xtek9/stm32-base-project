\documentclass[a4paper,10pt]{article}
\usepackage[utf8]{inputenc}
\usepackage{charter}
\usepackage[english]{babel}
\usepackage{amsmath}
\usepackage{amsfonts}
\usepackage{graphicx}
\usepackage{caption}
\usepackage{float}
\usepackage{hyperref}
\usepackage{setspace} 
\usepackage[top=2cm,bottom=2cm,left=3cm,right=2cm]{geometry}
\usepackage[usenames,dvipsnames]{xcolor}
\usepackage{setspace}
\usepackage{fixltx2e}     % to get subscript',
\usepackage[style=numeric,backend=biber]{biblatex}
\usepackage[acronym]{glossaries}
\usepackage{makeidx}
\usepackage{nomencl}      %Variable nomenclature
\usepackage{textcomp}
\usepackage{sfmath}
\usepackage{booktabs}     %to get bottomrule top rule and middlerun in tabular',
\usepackage{algorithm}
\usepackage{algorithmic}
\usepackage{rotating}

\title{Requirements}
\author{Foo Ever}
\addbibresource{./library.bib} %define so to work with mendeley',
\setlength{\parindent}{0pt}                % Avoid indentation on the first line',
\makeindex
\makenomenclature
\makeglossaries

\begin{document}
\maketitle
\begin{abstract}
This is abstract
\end{abstract}
\printnomenclature
\printglossaries

\nomenclature{$a$}{The variable a}

\section{The first section}
\newacronym[longplural={First Acronyms}]{FAlabel}{FA}{First Acronym}
This is how to use the \gls{FAlabel}. And now making a second use of \gls{FAlabel}.
\cite{fooAuthor}

\begin{figure}[ht]
	\centering
	\includegraphics[width=0.75 \textwidth]{resources/project_or_company_logo.png}
	\caption{caption}
	\label{reference}
\end{figure}

The first equation is referenced as \ref{eq:eqFirstequation}.
\begin{equation}
S~=~\Pi R^2
\label{eq:eqFirstequation}
\end{equation}

\printbibliography
\end{document}
